\documentclass[a4paper,11pt]{report}
\usepackage[T1]{fontenc}
\usepackage[utf8]{inputenc}
\usepackage{lmodern}
\usepackage[francais]{babel}

\title{}
\author{}

\begin{document}

\maketitle
\tableofcontents

\begin{abstract}
\end{abstract}

\chapter{Mise en œuvre}

\section{Réseau}

\subsection{Langage utilisé}
Le langage Java est utilisé pour le serveur et le client pour sa simplicité. En effet java permet un échange des données extrêmement simple grâce à la sérialisation de ses objets. De plus la création des connexions réseau se fait rapidement.
En utilisant ce langage, la communication entre les programmes a donc pu être réalisée rapidement, permettant aux développeurs de se consacrer davantage sur les protocoles.

\subsection{Bibliothèque utilisée}
Les bibliothèques utilisées pour le serveur sont :
\begin{enumerate}
  \item GraphStream, pour la visualisation de l'état du serveur sous forme de graphe. Car cette librairie est très similaire à la librairie PHP que nous avons utilisée aux parts avant.
  \item MySQL Connectors, pour la communication avec la base de donné. Car cette librairie contient de nombreux outils compatibles avec Java Swing.
  \item Java Zoom, pour la lecture de fichiers audio. Car cette librairie est très simple d’utilisation et contient une bonne documentation.
\end{enumerate}

\subsection{Test unitaire}
Pour pouvoir identifier rapidement les erreurs éventuelles dans les protocoles, ainsi que des failles de sécurité, des tests unitaires ont été réalisés. Ces tests ont été réalisées avec la librairie JUNIT. 



\end{document}
