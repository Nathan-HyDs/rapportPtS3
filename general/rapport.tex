% \documentclass[a4paper,11pt]{report}
% \usepackage[T1]{fontenc}
% \usepackage[utf8]{inputenc}
% \usepackage{lmodern}
% \usepackage[francais]{babel}
% \usepackage{graphicx}
% \usepackage{array}

\documentclass[a4paper, titlepage]{livret}

\usepackage[utf8]{inputenc} % accents
\usepackage[T1]{fontenc}      % caractères français
\usepackage{geometry}         % marges
\usepackage[francais]{babel}  % langue
\usepackage{graphicx}         % images
\usepackage{verbatim}         % texte préformaté
\usepackage{array}
\usepackage{graphicx}


\setlength{\parindent}{0cm}
\setlength{\parskip}{1ex plus 0.5ex minus 0.2ex}
\newcommand{\hsp}{\hspace{20pt}}
\newcommand{\HRule}{\rule{\linewidth}{0.5mm}}

\title{Data wars }

\author{Guillaume LAROYENNE, Nathan PRETOT \\ Jeremy RENAUD, Tom SALVI, Pierre VALENZA}

%
%
% \title{Rapport de stage}      % renseigne le titre
% \author{Prénom Nom}           %   "   "   l'auteur
% \date{18 juin 2007}           %   "   "   la future date de parution
%
%          % affiche un rappel discret (en haut à gauche)
%                               % de la partie dans laquel on se situe
%


\begin{document}
\begin{titlepage}
  \begin{sffamily}
  \begin{center}

    % Upper part of the page. The '~' is needed because \\
    % only works if a paragraph has started.
  %  \includegraphics[scale=0.04]{Assets/backOff.png}~\\[1.5cm]

    \textsc{\LARGE IUT informatique de Belfort-Montbéliard}\\[2cm]

    \textsc{\Large Rapport de projet tuteuré}\\[1.5cm]

    % Title
    \HRule \\[0.4cm]
    { \huge \bfseries Data wars\\[0.4cm] }

    \HRule \\[2cm]
    \includegraphics[scale=0.4]{../mainPageimg.png}
    \\[2cm]

    % Author and supervisor
    \begin{minipage}{0.4\textwidth}
      \begin{flushleft} \large
        Guillaume \textsc{LAROYENNE}, \\ Nathan \textsc{PRETOT}, \\ Jeremy \textsc{RENAUD}, \\ Tom \textsc{SALVI}, \\ Pierre \textsc{VALENZA}
      \end{flushleft}
    \end{minipage}
    \begin{minipage}{0.4\textwidth}
      \begin{flushright} \large
        \emph{Tutrice :} Mme. \textsc{Deschinkel} \\
      \end{flushright}
    \end{minipage}

    \vfill

    % Bottom of the page
    {\large 1\ier{} Octobre 2016 — 27 Mars 2016}

  \end{center}
  \end{sffamily}
\end{titlepage}





\maketitle
\tableofcontents


\chapter{Introduction}

	Le projet de deuxième année de DUT informatique est important. Le choix du sujet ainsi que l’équipe de projet le sont aussi. Outre le fait que le projet constitue une note importante pour notre moyenne du semestre, ce dernier peut être un plus pour notre avenir professionnel, ou même pour notre stage. Le choix du sujet, même si il a été plutôt rapide à trouver, a été très long à définir entièrement, en effet fixer les grandes lignes du sujet a été aisé : nous voulions un jeu de plateau, se jouant avec des cartes, et jouable en réseau. Cependant, pour délimiter entièrement le sujet il nous a fallut un temps conséquent qui s'explique par la création d'une multitude de règles pour rendre le jeu intéressant et attractif. De notre point de vue, ce projet avait donc un potentiel très intéressant car nous pouvions appliquer nos connaissances dans de nombreux domaines : en effet il mêle du réseau, de la base de donnée ainsi que du java. Mais également du PHP puisque nous avons dû réaliser un site web.
	
	L'équipe de projet est constituée de Guillaume LAROYENNE, de Nathan PRETOT, de Jérémy RENAUD, de Tom SALVI et de Pierre VALENZA.
	
	Ce projet a été encadré par Karine DESCHINKEL, enseignante à l'IUT de Belfort-Montbéliard. 

\chapter{Mise en œuvre}
  \section{Logiciel et méthode utilisé}
  \subsection{Organisation du projet}
     \begin{description}
       \item[Logiciel de gestion de versions :] nous avons utilisé \textit{git} et la plateforme \textit{GitHub} pour gérer le partage des fichiers et la gestion des versions. Nous avons utilisé ceci car, l'ayant appris en cours, tout les membres du projet étaient capable de l'utiliser.
       \item[Communication :] pour la communication d'information importante et d'information durable, l'utilisation de mail ou encore de système de discussion instantanée tel que \textit{messenger}. En plus de tout ceci, l'utilisation du logiciel de discussion \textit{Discord} nous a permit de parler du projet de vive voix.
       \item[Répartition des tâches :] pour la répartition des tâches et l'organisation du projet, nous nous sommes servis du site \textit{Trello}. Son interface simple, mais efficace, nous a permit de suivre l'avancé du projet.
     \end{description}
  
  \subsection{Jeu}
    \begin{description}
      \item[Langage de programmation :] \textit{Java} fut choisit pour plusieurs raisons, notamment, étant le langage le plus utilisés lors de notre formation, tout les membres de notre groupe avait un niveau équivalent sur celui-ci. De plus, la création d'une interface graphique en \textit{Java}, ayant été vue en cours, était donc connu par notre groupe.   
      \item[Base de données :] nous avons décidé d'utiliser \textit{MySQL}. Nous avons fait ce choix pour la simplicité d’intégration avec \textit{java} via la bibliothèque \textit{mysql-connector}.
      
    \end{description}
  
    \subsection{Interface}
    \subsubsection{Le Paradigme MVC}
    Nous avons structuré notre projet avec le paradigme MVC ( Modèle, Vue, Contrôleur), car ce format s'adapte parfaitement au jeu en tour par tour. En effets il permet de juste envoyer la partie modèle avec le réseau sans toucher ni aux contrôleurs ni à la vue.   
    \subsubsection{La bibliothèque graphique Java Swing}
    Nous avons décidé d'utiliser la bibliothèque graphique Java Swing car nous l'avions déjà utiliser sur plusieurs projets et que c'est la technologie que nous maitrisions le plus.
    \subsubsection{Logiciels}
    Voici les différents logiciels utilisés pour réaliser les graphismes.

    \begin{description}
      \item[Gimp :] Nous avons utilisé ce logiciel pour la modification des images de la vue. Nous avons choisi ce logiciel car c'est un logiciel gratuit et que l'on maitrisait déjà un peu. 
      \item[Inkscape :] Nous avons utilisé ce logiciel pour la création des images de la vue. Nous avons choisi inkscape car c'est un logiciel libre qui permet de faire du dessin vectoriel, et donc de pouvoir facilement redimensionner les images. 
    \end{description}



\subsection{Réseau}

\subsubsection{Langage utilisé}
Le langage Java est utilisé pour le serveur et le client pour sa simplicité. En effet java permet un échange des données extrêmement simple grâce à la sérialisation de ses objets. De plus la création des connexions réseau se fait rapidement.
En utilisant ce langage, la communication entre les programmes a donc pu être réalisée rapidement, permettant aux développeurs de se consacrer davantage sur les protocoles.

\subsubsection{Bibliothèque utilisée}
Les bibliothèques utilisées pour le serveur sont :
\begin{enumerate}
  \item GraphStream, pour la visualisation de l'état du serveur sous forme de graphe. Car cette librairie est très similaire à la librairie PHP que nous avons utilisée aux parts avant.
  \item MySQL Connectors, pour la communication avec la base de donné. Car cette librairie contient de nombreux outils compatibles avec Java Swing.
  \item Java Zoom, pour la lecture de fichiers audio. Car cette librairie est très simple d’utilisation et contient une bonne documentation.
\end{enumerate}

\subsubsection{Test unitaire}
Pour pouvoir identifier rapidement les erreurs éventuelles dans les protocoles, ainsi que des failles de sécurité, des tests unitaires ont été réalisés. Ces tests ont été réalisées avec la librairie JUNIT. 


	

  \section{Répartition du travail}
    Ce projet mettant en œuvre une grande diversité de domaine (site web, jeu, serveur, base de données, graphisme), nous n'avons donc pas eu d'autres choix que de répartir notre effectif sur les différents domaines. Laissant ainsi presque tout les domaines (à part modèle) couvert par une seule personne, nous laissant la disposition suivante :
    \begin{description}
      \item[Jeremy : ] implémentation des interfaces et des contrôleurs.
      \item[Tom : ] travail sur une partie du modèle du jeu.
      \item[Guillaume :] modélisation et création du serveur et des communications réseaux.
      \item[Pierre :] création du site web.
      \item[Nathan :] modélisation et création de la base de données. Modélisation du jeu et réalisation d'une partie de celui-ci.
    \end{description}
    
    \begin{figure}[th]
      \begin{center}
        \includegraphics[scale=0.365]{Assets/UMLRepartition.png}
        \caption{Répartition du travail}
        \label{RepTravail}
      \end{center}
    \end{figure}
    
    De plus des éléments du domaine informatiques, notre projet nous a demandé la création entière d'un jeu, nous demandant une réflexion autours des règles et fonctionnalités de celui-ci, ainsi que de créer des cartes innovantes pour celui-ci. Cette partie à été endossée par l'ensemble de l'équipe.
    
    	\subsection{Site internet}

 	\textbf{Langage de programmation :} PHP est un langage simple de programmation qui est le plus utilisé pour le développement de sites web. Nous l'avons aussi utilisé afin d'exploiter une base de données.

	\textbf{Framework :} nous avons choisi d'exploiter un micro-framework : Silex. Du fait qu'il soit minimaliste il laisse beaucoup de liberté aux développeurs comme le fait de ne pas imposer une architecture. De plus il offre un ensemble de services destiner aux contrôles de données.

	\textbf{Framework CSS :} Bootstrap est un kit CSS créer par les développeurs de Twitter. Il permet de développer un site reponsive, donc pouvant s'adapter sur différents appareils, facilement et efficacement.

    
  \section{Gestion du temps}
    Le projet a débuté sur la conception du jeu. C'est-à-dire la création des règles, ainsi que les éléments de modélisation au niveau informatique ( MCD, UML, etc...).  C'est étape, qui a durée un mois et demis, fut suivit du début de la phase de programmation du réseau et du jeu. La partie interface commença un peu plus tard, car elle nécessitais d'avoir une partie du modèle de terminée pour commencer. Le site web ne commença qu'a partir de janvier car celui-ci n'était pas prévus dans le travail initiale. Une première version du jeu en mono-poste fut terminée au milieu des vacances de février, suivis de près par la version réseau.
    \begin{figure}[th]
      \begin{center}
        \includegraphics[scale=0.5]{Assets/gestionTemps.png}
        \caption{Gestion du temps par rapport à la courbe des \textit{commits} \textit{GitHub}}
        \label{RepTime}
      \end{center}
    \end{figure}
  


\section{Rendu}
\subsection{Les composants du projet}
A ce jour, le projet comporte :

	\subsubsection{Un site web permettant :}
	
	 - De créer un compte utilisateur
		 
	 - De créer et modifier son deck de carte pour le jouer en jeu.
	 
	 - De télécharger le jeu.
	 \begin{figure}[th]
	 \begin{center}
	  \includegraphics[scale=0.2]{Assets/site.png} 
	\caption{Page de création du deck}
     \label{siteWeb}
      \end{center}
    \end{figure}

	
	\subsubsection{Un système réseau avec interface réutilisable\\ sur d'autres 			projets :}
	
	Il permet de créer un groupe, envoyer une invitation au membres connectés, ces derniers peuvent gérer les invitations reçus ( accepter ou non ) l'invitation est limitée dans le temps pour éviter d'attendre un joueur éternellement. Une fois le groupe créé l'hôte peut lancer la partie.   
	
	 \begin{figure}[th]
	\begin{center}
	\includegraphics[scale=0.5]{Assets/connection3.png} 
	\caption{Interface du module réseau}
     \label{interface reseau}
      \end{center}
    \end{figure}

	
	\subsubsection{Un jeu jouable en Réseau de 2 à 4 joueurs.}
	
	Dans le jeu on peut récupérer le deck en fonction du joueur\\
	 \begin{figure}[th]	
	\begin{center}
	\includegraphics[scale=0.3]{Assets/mainJoueur.png} 
	\caption{main du joueur}
     \label{cartesJoueur}
      \end{center}
    \end{figure}

	
	 Jouer des cartes
	
	 Déplacer des unités
	  \begin{figure}[th]
	 \begin{center}
	\includegraphics[scale=1]{Assets/UniteeSelectMove.png} 
	\caption{déplacement d'une unité}
     \label{MouvementUnité}
      \end{center}
    \end{figure}

	 
	Attaquer des unités ennemies
	
	Capturer des bâtiments de ressources
	
	 \begin{figure}[th]
	\begin{center}
	\includegraphics[scale=0.5]{Assets/CaptureSucces.png} 
	\caption{Capture d'un bâtiment}
     \label{CaptureBatiment}
      \end{center}
    \end{figure}

	
	Voir les informations des entités sur la carte
	 \begin{figure}[th]
	\begin{center}
	\includegraphics[scale=0.3]{Assets/infosCarte.png} 
	\caption{Informations sur une unité}
     \label{InfoUnité}
      \end{center}
    \end{figure}

	
    La gestion des ressources est fonctionnelle
	
	 On peut attaquer le QG ennemi pour gagner la partie.
	
 La partie de chaque joueur est synchronisée à chaque fin de tour d'un joueur.
	
	


  
  \chapter{Bilan}
    \section{Bilan pédagogique}
      Ce projet nous a permit de renforcer nos acquis dans des langages tels que \textit{Java}, \textit{MySQL} ou encore \textit{PHP} avec le \textit{framework} <<silex>>. Mais ce projet nous également permit d'acquérir de nouvelle connaissance, notamment comment lié \textit{Mysql} a \textit{Java}, ou encore la communication réseaux en \textit{Java}. Ce projet, et quelques mésaventures avec des modifications de code altérant le bon fonctionnement du programme, nous a appris que la pratique du test logiciel était importante pour la programmation d'un jeu.
      
    \section{Bilan technique}
       Nous avons à l'heure actuelle un programme fonctionnel malgré quelques failles notamment dû à un manque de rigueur dans la gestion des contrôleurs, mais également de quelques problèmes avec les effets. Certains de ceux-ci ne s'appliquent pas. Si notre programme reste viable dans le fond, la forme reste encore à améliorer. En effet, toutes les cartes n'ont pas encore d'images et l'interface générale du reste minime. Les points évoqués précédemment correspondent aux potentielles voix d'améliorations du projet.

\end{document}
