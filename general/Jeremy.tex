\documentclass[a4paper,11pt]{report}
\usepackage[T1]{fontenc}
\usepackage[utf8]{inputenc}
\usepackage{lmodern}
\usepackage[francais]{babel}
\usepackage{graphicx}
\usepackage{array}

\title{Data wars}
\author{Guillaume LARROYENNE, Nathan PRETOT, Jeremy Renaud,Tom SALVI, Pierre VALENZA}

\begin{document}

\maketitle
\tableofcontents

\begin{abstract}
\end{abstract}

\chapter{Mise en Oeuvre}



\section{Rendu}
\subsection{Les composants du projet}
A ce jour, le projet comporte :

	\subsubsection{Un site web permettant :}
	
	 - De créer un compte utilisateur
		 
	 - De créer et modifier son deck de carte pour le jouer en jeu.
	 
	 - De télécharger le jeu.
	 
	 \begin{center}
	\includegraphics[scale=0.2]{Assets/site.png} 
	\end{center}

	
	\subsubsection{Un système réseau avec interface réutilisable\\ sur d'autres 			projets :}
	
	Il permet de créer un groupe, envoyer une invitation au membres connectés, ces derniers peuvent gérer les invitations reçus ( accepter ou non ) l'invitation est limitée dans le temps pour éviter d'attendre un joueur éternellement. Une fois le groupe créé l'hôte peut lancer la partie.   
	
	
	\begin{center}
	\includegraphics[scale=0.5]{Assets/connection3.png} 
	\end{center}
	
	\subsubsection{Un jeu jouable en Réseau de 2 à 4 joueurs.}
	
	Dans le jeu on peut récupérer le deck en fonction du joueur\\
	\begin{center}
	\includegraphics[scale=0.3]{Assets/mainJoueur.png} 
	\end{center}
	
	 Jouer des cartes
	
	 Déplacer des unités
	 \begin{center}
	\includegraphics[scale=1]{Assets/UniteeSelectMove.png} 
	\end{center}
	 
	Attaquer des unités ennemies
	
	Capturer des bâtiments de ressources
	
	\begin{center}
	\includegraphics[scale=0.5]{Assets/CaptureSucces.png} 
	\end{center}
	
	Voir les informations des entités sur la carte
	
	\begin{center}
	\includegraphics[scale=0.3]{Assets/InfosCarte.png} 
	\end{center}
	
    La gestion des ressources est fonctionnelle
	
	 On peut attaquer le QG ennemi pour gagner la partie.
	
 La partie de chaque joueur est synchronisée à chaque fin de tour d'un joueur.
	
	


	

\end{document}
