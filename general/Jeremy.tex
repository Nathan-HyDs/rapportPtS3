\documentclass[a4paper,11pt]{report}
\usepackage[T1]{fontenc}
\usepackage[utf8]{inputenc}
\usepackage{lmodern}
\usepackage[francais]{babel}
\usepackage{graphicx}
\usepackage{array}

\title{Data wars}
\author{Guillaume LARROYENNE, Nathan PRETOT, Jeremy Renaud,Tom SALVI, Pierre VALENZA}

\begin{document}

\maketitle
\tableofcontents

\begin{abstract}
\end{abstract}

\chapter{Mise en Oeuvre}
\section{Interface}
\subsection{Le Paradigme MVC}
Nous avons structuré notre projet avec le paradigme MVC ( Modèle, Vue, Contrôleur), car ce format s'adapte parfaitement au jeu en tour par tour. En effets il permet de juste envoyer la partie modèle avec le réseau sans toucher ni aux contrôleurs ni à la vue.   
\subsection{La bibliothèque graphique Java Swing}
Nous avons décidé d'utiliser la bibliothèque graphique Java Swing car nous l'avions déjà utiliser sur plusieurs projets et que c'est la technologie que nous maitrisions le plus.
\subsection{Logiciels}
Voici les différents logiciels utilisés pour réaliser les graphismes.
\subsubsection{Gimp}
Nous avons utilisé ce logiciel pour la modification des images de la vue. Nous avons choisi ce logiciel car c'est un logiciel gratuit et que l'on maitrisait déjà un peu. 
\subsubsection{Inkscape}
Nous avons utilisé ce logiciel pour la création des images de la vue. Nous avons choisi inkscape car c'est un logiciel libre qui permet de faire du dessin vectoriel, et donc de pouvoir facilement redimensionner les images. 
\section{Rendu}
\subsection{Les composants du projet}
A ce jour, le projet comporte :

	\subsubsection{Un site web permettant :}
	
	 - De créer un compte utilisateur
		 
	 - De créer et modifier son deck de carte pour le jouer en jeu.
	 
	 - De télécharger le jeu.
	 
	 \begin{center}
	\includegraphics[scale=0.2]{Assets/site.png} 
	\end{center}

	
	\subsubsection{Un système réseau avec interface réutilisable\\ sur d'autres 			projets :}
	
	Il permet de créer un groupe, envoyer une invitation au membres connectés, ces derniers peuvent gérer les invitations reçus ( accepter ou non ) l'invitation est limitée dans le temps pour éviter d'attendre un joueur éternellement. Une fois le groupe créé l'hôte peut lancer la partie.   
	
	
	\begin{center}
	\includegraphics[scale=0.5]{Assets/connection3.png} 
	\end{center}
	
	\subsubsection{Un jeu jouable en Réseau de 2 à 4 joueurs.}
	
	Dans le jeu on peut récupérer le deck en fonction du joueur, jouer des cartes, déplacer des unités, attaquer des unités ennemies, capturer des bâtiments de ressources, voir les informations des entités sur la carte. La gestion des ressources est fonctionnelle.
	
	 On peut attaquer le QG ennemi pour gagner la partie.
	
 La partie de chaque joueur est synchronisée à chaque fin de tour d'un joueur.
	\begin{center}
	\includegraphics[scale=0.3]{Assets/UniteeSelectMove.png} 
	\end{center}
	


	

\end{document}
